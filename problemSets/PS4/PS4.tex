\documentclass[12pt,letterpaper]{article}
\usepackage{graphicx,textcomp}
\usepackage{natbib}
\usepackage{setspace}
\usepackage{fullpage}
\usepackage{color}
\usepackage[reqno]{amsmath}
\usepackage{amsthm}
\usepackage{fancyvrb}
\usepackage{amssymb,enumerate}
\usepackage[all]{xy}
\usepackage{endnotes}
\usepackage{lscape}
\newtheorem{com}{Comment}
\usepackage{float}
\usepackage{hyperref}
\newtheorem{lem} {Lemma}
\newtheorem{prop}{Proposition}
\newtheorem{thm}{Theorem}
\newtheorem{defn}{Definition}
\newtheorem{cor}{Corollary}
\newtheorem{obs}{Observation}
\usepackage[compact]{titlesec}
\usepackage{dcolumn}
\usepackage{tikz}
\usetikzlibrary{arrows}
\usepackage{multirow}
\usepackage{xcolor}
\newcolumntype{.}{D{.}{.}{-1}}
\newcolumntype{d}[1]{D{.}{.}{#1}}
\definecolor{light-gray}{gray}{0.65}
\usepackage{url}
\usepackage{listings}
\usepackage{color}

\definecolor{codegreen}{rgb}{0,0.6,0}
\definecolor{codegray}{rgb}{0.5,0.5,0.5}
\definecolor{codepurple}{rgb}{0.58,0,0.82}
\definecolor{backcolour}{rgb}{0.95,0.95,0.92}

\lstdefinestyle{mystyle}{
	backgroundcolor=\color{backcolour},   
	commentstyle=\color{codegreen},
	keywordstyle=\color{magenta},
	numberstyle=\tiny\color{codegray},
	stringstyle=\color{codepurple},
	basicstyle=\footnotesize,
	breakatwhitespace=false,         
	breaklines=true,                 
	captionpos=b,                    
	keepspaces=true,                 
	numbers=left,                    
	numbersep=5pt,                  
	showspaces=false,                
	showstringspaces=false,
	showtabs=false,                  
	tabsize=2
}
\lstset{style=mystyle}
\newcommand{\Sref}[1]{Section~\ref{#1}}
\newtheorem{hyp}{Hypothesis}

\title{Problem Set 4}
\date{Due: April 4, 2022}
\author{Applied Stats II}


\begin{document}
	\maketitle
	\section*{Instructions}
	\begin{itemize}
		\item Please show your work! You may lose points by simply writing in the answer. If the problem requires you to execute commands in \texttt{R}, please include the code you used to get your answers. Please also include the \texttt{.R} file that contains your code. If you are not sure if work needs to be shown for a particular problem, please ask.
		\item Your homework should be submitted electronically on GitHub in \texttt{.pdf} form.
		\item This problem set is due before class on Monday April 4, 2022. No late assignments will be accepted.
		\item Total available points for this homework is 80.
	\end{itemize}

	\vspace{.25cm}
\section*{Question 1}
\vspace{.25cm}
\noindent We're interested in modeling the historical causes of infant mortality. We have data from 5641 first-born in seven Swedish parishes 1820-1895. Using the "infants" dataset in the \texttt{eha} library, fit a Cox Proportional Hazard model using mother's age and infant's gender as covariates. Present and interpret the output.

\lstinputlisting[language=R, firstline=31, lastline=44]{PS4.R}

First, we assign a survival object for the children using entry, exit, and event data to the age of 15. We are then able to get a graph of the total population's survival rate. As seen below. 
 
\includegraphics{Rplot1.pdf}

We are also able to graph the difference in survival rates between males and females.

\includegraphics{Rplot3.pdf} 

The next stage is to fit a Cox Proportional Hazard Model, which we do with the following code. 

\lstinputlisting[language=R, firstline=51, lastline=54]{PS4.R}

There is a 0.0822 decrease in the expected log of the hazard for female babies compared to male babies, holding m.age constant. There is a 0.0076 increase in the expected log of the hazard for babies of mothers with one extra unit of age, holding sex constant. We also note that m.age has a p-value that is statistically significant, as is the p-vlaue for gender. 

% Table created by stargazer v.5.2.3 by Marek Hlavac, Social Policy Institute. E-mail: marek.hlavac at gmail.com
% Date and time: Mon, Apr 04, 2022 - 13:05:49
\begin{table}[!htbp] \centering 
	\caption{} 
	\label{} 
	\begin{tabular}{@{\extracolsep{5pt}}lc} 
		\\[-1.8ex]\hline 
		\hline \\[-1.8ex] 
		& \multicolumn{1}{c}{\textit{Dependent variable:}} \\ 
		\cline{2-2} 
		\\[-1.8ex] & enter \\ 
		\hline \\[-1.8ex] 
		sexfemale & $-$0.082$^{***}$ \\ 
		& (0.027) \\ 
		& \\ 
		m.age & 0.008$^{***}$ \\ 
		& (0.002) \\ 
		& \\ 
		\hline \\[-1.8ex] 
		Observations & 26,574 \\ 
		R$^{2}$ & 0.001 \\ 
		Max. Possible R$^{2}$ & 0.986 \\ 
		Log Likelihood & $-$56,503.480 \\ 
		Wald Test & 22.520$^{***}$ (df = 2) \\ 
		LR Test & 22.518$^{***}$ (df = 2) \\ 
		Score (Logrank) Test & 22.530$^{***}$ (df = 2) \\ 
		\hline 
		\hline \\[-1.8ex] 
		\textit{Note:}  & \multicolumn{1}{r}{$^{*}$p$<$0.1; $^{**}$p$<$0.05; $^{***}$p$<$0.01} \\ 
	\end{tabular} 
\end{table} 

\lstinputlisting[language=R, firstline=61, lastline=61]{PS4.R}

The hazard ratio of female babies is 0.9210739 that of male babies. This means that for every 100 male babies that die, we would expect that only ~ 92 female babies would die. This could be interrupted in another sense as female babies have a 8 percent lower death rate than equivalent male babies.

\lstinputlisting[language=R, firstline=66, lastline=66]{PS4.R}
The hazard ratio of a mother aged 0 is 1.0076466, this number is not quite accurate as no mother can have a child a age zero but it does allow us to analyse a one unit increase in age, and its affect on infant mortality. This could be interrupted in another sense as for each unit increase in a mothers age, the chance of that child dying is 0.7 percent higher than if the mother was one unit of age younger. 

\lstinputlisting[language=R, firstline=71, lastline=72]{PS4.R}

We now graph the two covariets in our model. Which shows that as age increase the bounds of certinly also expand.  

\includegraphics{Rplot4.pdf}

We now will run an interaction model to compare with additive model.

\lstinputlisting[language=R, firstline=82, lastline=86]{PS4.R}
 
There is a 0.127105 decrease in the expected log of the hazard for female babies compared to male babies, holding m.age constant.

There is a 0.006963 increase in the expected log of the hazard for babies of mothers with one extra unit of age, holding sex constant. However the P-Values are at level that they are not statistically significant when compared to the additive model.

% Table created by stargazer v.5.2.3 by Marek Hlavac, Social Policy Institute. E-mail: marek.hlavac at gmail.com
% Date and time: Mon, Apr 04, 2022 - 13:20:59
\begin{table}[!htbp] \centering 
	\caption{} 
	\label{} 
	\begin{tabular}{@{\extracolsep{5pt}}lc} 
		\\[-1.8ex]\hline 
		\hline \\[-1.8ex] 
		& \multicolumn{1}{c}{\textit{Dependent variable:}} \\ 
		\cline{2-2} 
		\\[-1.8ex] & child\_surv \\ 
		\hline \\[-1.8ex] 
		sexfemale & $-$0.127 \\ 
		& (0.140) \\ 
		& \\ 
		m.age & 0.007$^{**}$ \\ 
		& (0.003) \\ 
		& \\ 
		sexfemale:m.age & 0.001 \\ 
		& (0.004) \\ 
		& \\ 
		\hline \\[-1.8ex] 
		Observations & 26,574 \\ 
		R$^{2}$ & 0.001 \\ 
		Max. Possible R$^{2}$ & 0.986 \\ 
		Log Likelihood & $-$56,503.430 \\ 
		Wald Test & 22.530$^{***}$ (df = 3) \\ 
		LR Test & 22.624$^{***}$ (df = 3) \\ 
		Score (Logrank) Test & 22.562$^{***}$ (df = 3) \\ 
		\hline 
		\hline \\[-1.8ex] 
		\textit{Note:}  & \multicolumn{1}{r}{$^{*}$p$<$0.1; $^{**}$p$<$0.05; $^{***}$p$<$0.01} \\ 
	\end{tabular} 
\end{table} 

\lstinputlisting[language=R, firstline=92, lastline=86]{PS4.R}

The hazard ratio of a mother aged 0 is 0.8806412 that of male babies. This means that for every 100 male babies that die, we would expect that only ~ 88 female babies would die. This could be interrupted in another sense as female babies have a 8 percent lower death rate than equivalent male babies.

\lstinputlisting[language=R, firstline=97, lastline=86]{PS4.R}

The hazard ratio of a mother aged 0 is 1.006987, this number is not quite accurate as no mother can have a child a age zero but it does allow us to analyse a one unit increase in age, and its affect on infant mortality. This could be interrupted in another sense as for each unit increase in a mothers age, the chance of that child dying is 0.6987 percent higher than if the mother was one unit of age younger. 
\vspace{.25cm}


While the interactive model had a Chisq value of 0.10623, and the additive has a value of 9.464. The P values in the additive model were statisically significant and better explain the covariets affect on child mortality in this case. 

\end{document}
